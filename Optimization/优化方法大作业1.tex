\documentclass[12pt,a4paper,UTF-8]{article}
\pagestyle{plain}
\usepackage{fontspec,ctex,amsthm, amsmath, geometry, listingsutf8, graphicx, enumerate, multirow}
\usepackage{listings}
\usepackage[framed,numbered,autolinebreaks,useliterate]{mcode}
\title{ 优化方法基础大作业 }
\author{姓名:张和永 / 班级:计算机54班 / 学号:2150500113}
\date{\today}

\begin{document}
\maketitle
\setcounter{tocdepth}{4}
\tableofcontents
\newpage
\section{实验题目}

对于$R^2$空间非二次规划问题$$min\ f(x) = e^{x_1 + x_2 - 0.1} + e^{x_1 - 3x_2 -0.1} + e^{-x_1 - 0.1}$$分析回溯直线搜索采用相同的初始值和不同的$\alpha, \beta$时, 误差迭代次数改变的情况。

\section{实验原理}

\subsection{梯度下降法}

\begin{enumerate}

	\item 给定初始点$x^0 \in dom\ f$, 令$k=0$ 

	\item 判断是否停止:如果$\lVert \nabla f(x^k) \rVert \leq \epsilon$, 停止 

	\item 使用梯度下降法计算$x^k$处的下降方向:确定$d^k$满足$\exists \hat{t} > 0 \Rightarrow f(x^k + t d^k) < f(x^k), \forall t \in (0, \hat{t})$

	\begin{itemize}

		\item[] $d^k = -\nabla f(x^k) $

	\end{itemize}

	\item 直线搜索:确定$t^k > 0$满足$f(x^k + t^k d^k) < f(x^k)$, 更新$x^{k+1} = x^k, k = k + 1$, 回2


\end{enumerate}


\subsection{回溯直线搜索}

\begin{enumerate}

	\item 给定初始点$x^0 \in dom\ f$, 令$k=0$ 

	\item 判断是否停止:如果$\lVert \nabla f(x^k) \rVert \leq \epsilon$, 停止 

	\item 使用梯度下降法计算$x^k$处的下降方向:确定$d^k$满足$\exists \hat{t} > 0 \Rightarrow f(x^k + t d^k) < f(x^k), \forall t \in (0, \hat{t})$

	\begin{itemize}

		\item[] $d^k = -\nabla f(x^k) / \lVert f(x^k) \rVert$

	\end{itemize}

	\item 使用回溯直线搜索确定$t^k > 0$满足$f(x^k + t^k d^k) < f(x^k)$, 令$x^{k+1} = x^k, k = k + 1$, 回2

	\begin{enumerate}[(1)]

		\item 给定$f$在$x \in dom\ f$处的下降方向$d^k$, 参数$0 < \alpha < 0.5, 0 < \beta < 1$

		\item $t = 1$

		\item while $f(x + t d^k) > f(x) + \alpha t \nabla f(x)^T d^k$: $t = \beta t$ 

	\end{enumerate}

\end{enumerate}

\newpage

\section{MATLAB源程序}
\begin{lstlisting}
syms x y;
v = [x;y];
f = exp(x+3*y-0.1)+exp(x-3*y-0.1)+exp(-x-0.1);%原函数
grad=jacobian(f,v);%梯度函数
alpha = 0;
beta = 0;
while beta < 0.8
	beta = beta + 0.2;
	while alpha < 0.4
		alpha = alpha + 0.1;
		cnt = 0;%计数
		cur = [20;20];%当前点
		cur_val = double(subs(f,v,cur));
		cur_grad = double(subs(grad,v,cur))';
		temp = norm(cur_grad);
		vec_val=[];
		vec_cnt=[];
		while temp > 10^(-2)
			step = 1;
			next = cur - step * cur_grad;
			cur_val = double(subs(f,v,cur));
			next_val = double(subs(f,v,next));
			vec_val = [vec_val cur_val];
			vec_cnt = [vec_cnt cnt];
			del = - alpha * (cur_grad' * cur_grad);
			del_temp = step * del;
			while next_val > cur_val + del_temp
				step = step * beta;
				next = cur - step * cur_grad;
				next_val = double(subs(f,v,next));
				del_temp = step * del;
			end
			cnt = cnt + 1;
			cur = next;
			cur_grad = double(subs(grad,v,cur))';
			temp = norm(cur_grad);
		end
		fprintf('a为%.1f,b为%.1f,迭代%d次,
		val=%f\n',alpha,beta,cnt,cur_val)      
		figure
		semilogy(vec_cnt,vec_val-cur_val,'k-')
		axis normal
		title(['a=',num2str(alpha),',b=',num2str(beta)])       
		xlabel('cnt') 
		ylabel('del')
	end
	alpha = 0;
end

\end{lstlisting}

\section{实验结果}

\subsection{运行结果} 

\subparagraph{解得}$x^{*} \approx (-0.3465736, 0),\ f(x^{*}) \approx 2.559266$

\subparagraph{不同的$\alpha, \beta$时误差迭代次数改变的情况\\}
\includegraphics[width=3in]{01.eps}
\includegraphics[width=3in]{02.eps}\\
\includegraphics[width=3in]{03.eps}
\includegraphics[width=3in]{04.eps}\\
\includegraphics[width=3in]{11.eps}
\includegraphics[width=3in]{12.eps}\\
\includegraphics[width=3in]{13.eps}
\includegraphics[width=3in]{14.eps}\\
\includegraphics[width=3in]{21.eps}
\includegraphics[width=3in]{22.eps}\\
\includegraphics[width=3in]{23.eps}
\includegraphics[width=3in]{24.eps}\\
\includegraphics[width=3in]{31.eps}
\includegraphics[width=3in]{32.eps}\\
\includegraphics[width=3in]{33.eps}
\includegraphics[width=3in]{34.eps}\\

\subparagraph{}综合以上图像可得出下表

\begin{table}[!hbp]
	\centering\begin{tabular}{|c|c|c|c|c|}
		\hline
		$Iterations$ & $\alpha=0.1$ & $\alpha=0.2$ & $\alpha=0.3$ & $\alpha=0.4$ \\
		\hline
		$\beta=0.2$ & 48 & 85 & 101 & 108 \\
		\hline
		$\beta=0.4$ & 32 & 48 & 66 & 69 \\
		\hline
		$\beta=0.6$ & 72 & 76 & 85 & 73 \\
		\hline
		$\beta=0.8$ & 68 & 81 & 94 & 96 \\
		\hline
	\end{tabular}
\end{table}  


\subsection{结果分析}

\begin{itemize}

	\item 当$\beta$不变时, 迭代次数随$\alpha$的增大而增大;

	\item 当$\alpha$不变时, $\beta$适中时迭代次数较少。
	\item 在初始点相同的情况下,$\alpha$和$\beta$的不同取值会对迭代次数有不同程度的影响,但总体上不会产生戏剧性的影响。
	

\end{itemize}

\end{document}