\documentclass[UTF8]{article}
\pagestyle{plain}
\usepackage{ctex,xcolor,amsthm, amsmath, geometry, graphicx, enumerate, multirow,amssymb}
\usepackage{listings}

\title{ \emph{电子技术实验报告}\\---电梯控制器的设计与分析}
\author{姓名:张和永 \\班级:计算机54班 \\ 学号:2150500113}
\date{\today}

\begin{document}
\maketitle
\setcounter{tocdepth}{10}
\tableofcontents
\newpage

\section{实验目的}
电子技术专题实验是对“数字逻辑”课程内容的全面、系统的总结、巩固和提高的一项课程实践活动。
根据数字逻辑的特点,选择相应的题目,在老师的指导下,由学生独立完成。目的是通过实验使学生
掌握数字逻辑电路设计的基本方法和技巧,正确运用QuartusⅡ软件及实验室多功能学习机硬件平台,
完成所选题目的设计任务,并掌握数字逻辑电路测试的基本方法,训练学生的动手能力和思维方法。
通过实验,一方面提高运用数字逻辑电路解决实际问题的能力,另一方面使学生更深入的理解所学知识,
为以后的计算机硬件课程的学习奠定良好的基础。

\section{项目设计概要}

\subsection{设计实现的目标}
  随着社会的发展,电梯的使用越来越普遍,已从原来只在商业大厦、宾馆使用,过渡到在办公楼、居民楼等场所使用,
  并且对电梯功能的要求也不断提高,相应地其控制方式也在不停地发生变化。对于电梯的控制,传统的方法是使用
  继电器—接触器控制系统进行控制,随着技术的不断发展,微型计算机在电梯控制上的应用日益广泛,现在已进入全微机化控制的时代。

\subsection{整体设计概述}
电梯的微机化控制主要有以下几种形式:
\begin{enumerate}
\item PLC控制
\item 单板机控制
\item 单片机控制
\item 单微机控制
\item 多微机控制
\item 人工智能控制
\end{enumerate}
随着EDA技术的快速发展,CPLD/FPGA已广泛应用于电子设计与控制的各个方面。
本设计就是使用一片CPLD/FPGA来实现对电梯的控制的.

\subsection{项目设计特点}
我们在项目设计过程中采用模块化设计思想,事先制定了模块间的接口方案,
使得整个系统的组合变得十分灵活。由于我们在设计时为电子钟和秒表模块中都加入了显示电路,
总控模块可分别与之连接组成一个分系统,便于调试。在最终整合时,
我们也只需要将两个模块中的显示电路合二为一即可。


\section{系统设计方案}
系统工作用2 Hz基准时钟信号CLKIN,楼层上升请求键UPIN,楼层下降请求键DOWNIN,
楼层选择键入键ST\_CH,提前关门输入键CLOSE,延迟关门输入键DELAY,电梯运行的开关键RUN\_STOP,
电梯运行或停止指示键LAMP,电梯运行或等待时间指示键RUN\_WAIT,电梯所在楼层指示数码管ST\_OUT,
楼层选择指示数码管DIRECT。楼层选择键入键ST\_CH,提前关门输入键CLOSE,延迟关门输入键DELAY,
电梯运行的开关键RUN\_STOP,电梯运行或停止指示键LAMP,电梯运行或等待时间指示键RUN\_WAIT,
电梯所在楼层指示数码管ST\_OUT,楼层选择指示数码管DIRECT。
\subsection{输入端口}
\begin{enumerate}
\item CLKIN:基准时钟信号,为系统提供2Hz的时钟脉冲,上升沿有效;
\item UPIN:电梯上升请求键。由用户向电梯控制器发出上升请求。高电平有效;
\item DOWNIN:电梯下降请求键,由用户向电梯控制器发出下降请求。高电平有效;
\item ST\_CH[2..0]:楼层选择键入键,结合DIRECT完成楼层选择的键入,高电平有效;
\item CLOSE:提前关门输入键。可实现无等待时间的提前关门操作,高电平有效;
\item DELAY:	延迟关门输入键。可实现有等待时间的延迟关门操作,高电平有效;
\item RUN\_STOP:电梯运行或停止开关键。可实现由管理员控制电梯的运行或停止,高电平有效。
\end{enumerate}
\subsection{输出端口}
\begin{enumerate}
\item LAMP:电梯运行或等待指示键,指示电梯的运行或等待状况。高电平有效;
\item RUN\_WAIT:电梯运行或等待时间指示键,指示电梯运行状况或等待时间的长短,高电平有效;
\item ST\_OUT:电梯所在楼层指示数码管,只是电梯当前所在的楼层数。即1~5层,高电平有效;
\item DIRECT:楼层选择指示数码管,指示用户所要选择的楼层数,高电平有效。
\end{enumerate}
\subsection{系统功能模块设计示意图}
\begin{center}
    \includegraphics[width=4in]{系统功能模块设计示意图.png}
\end{center}

\subsection{项目分块及其实现方案}
电梯的控制状态包括运行状态、停止状态及等待状态,其中运行状态又包含向上状态和向下状态。
主要动作有开、关门,停靠和启动。乘客可通过键入开、关门按钮,呼唤按钮,指定楼层按钮等来控制电梯的行与停。
据此,整个电梯控制器DTKZQ应包括如下几个组成部分:
\begin{enumerate}
\item 时序输出及楼选计数器
\item 电梯服务请求处理器
\item 电梯升降控制器
\item 上升及下降寄存器
\item 电梯次态生成器
\end{enumerate}

该电梯控制器设计的关键是确定上升及下降寄存器的置位与复位。整个系统的内部组成结构图如图2所示
\begin{center}
    \includegraphics[width = 4in]{电梯控制器的内部组成结构图.png}
\end{center}

\section{VHDL源程序}

\lstset{%
alsolanguage=VHDL,
tabsize=4, 
  frame=shadowbox, %把代码用带有阴影的框圈起来
  commentstyle=\color{red!50!green!50!blue!50},%浅灰色的注释
  rulesepcolor=\color{red!20!green!20!blue!20},%代码块边框为淡青色
  keywordstyle=\color{blue!90}\bfseries, %代码关键字的颜色为蓝色,粗体
  showstringspaces=false,%不显示代码字符串中间的空格标记
  stringstyle=\ttfamily, % 代码字符串的特殊格式
  keepspaces=true, %
  breakindent=22pt, %
  numbers=left,%左侧显示行号 往左靠,还可以为right,或none,即不加行号
  stepnumber=1,%若设置为2,则显示行号为1,3,5,即stepnumber为公差,默认stepnumber=1
  %numberstyle=\tiny, %行号字体用小号
  numberstyle={\color[RGB]{0,192,192}\tiny} ,%设置行号的大小,大小有tiny,scriptsize,footnotesize,small,normalsize,large等
  numbersep=8pt,  %设置行号与代码的距离,默认是5pt
  basicstyle=\footnotesize, % 这句设置代码的大小
  showspaces=false, %
  flexiblecolumns=true, %
  breaklines=true, %对过长的代码自动换行
  breakautoindent=true,%
  breakindent=4em, %
  escapebegin=\begin{CJK*}{GBK}{hei},escapeend=\end{CJK*},
  aboveskip=1em, %代码块边框
  tabsize=2,
  showstringspaces=false, %不显示字符串中的空格
  backgroundcolor=\color[RGB]{245,245,244}   %代码背景色
}

\lstinputlisting{DTKZQ.VHDL}


\section{系统仿真}
\subsection{模拟仿真测试波形图}
\begin{center}
    \includegraphics{系统仿真.png}
\end{center}

\subsection{DTKZQ器件图}
\begin{center}
    \includegraphics{DTKZQ器件图.png}
\end{center}

\section{项目总结}
通过本次课程设计进一步熟悉Quartus II软件的使用和操作方法,以及硬件实现时的下载方法与运行方法;
对Verilog HDL语言的自顶向下设计方法有了进一步的认识,对其中的许多语句也有了新了解,掌握;
在课程设计的过程中和小组成员充分地讨论与合作,提高了自身的程序设计能力与沟通交际能力;也充分了解了一个
项目从始至终的设计流程,深有感触。最后,十分有幸能参加到这次的设计课程,它使我受益匪浅。

\end{document}