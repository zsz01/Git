\documentclass{article}
\pagestyle{plain}
\usepackage{ctex, enumerate, multirow, graphicx, pythonhighlight}
\begin{document}
\tableofcontents
\newpage

\newpage
\section{软件说明}

\subsection{软件功能}
% Python版本,需要的第3方包或其他的集成环境等等。
% 这里只需列出需要的资源的名称和版本,并不需要提供软件本身。对于需要单独下载的软件环境,应提供获取方式,如网址、下载方法、下载哪个版本等信息。

此学籍管理系统用python语言实现,设计的功能包括:
\begin{itemize}
    \item 学生分数的修改
    \item 学生分数的查询
    \item 学生个人信息的添加
    \item 学生个人信息的删除
    \item 年级成绩的排序
    \item 年级成绩的图表
    \item 管理系统密码的修改
\end{itemize}

\subsection{运行环境}
\begin{itemize} 
\item Anaconda Version 5.0.1 Python 3.6 
\item Windows 10 64-bit operating system, x64-based processor
\end{itemize}

\subsection{环境搭建说明}
% 说明建立运行环境的步骤。读者能按此建立起能运行该软件的环境,再现运行结果。安装方法、安装位置、顺序、配置参数等。

\begin{enumerate}
    \item 从https://www.anaconda.com/download/\#windows 安装Anaconda Python 3.6
    \item 将src中'info.txt'和'学籍管理系统.py'复制到同一个文件夹目录下
    \item 在该目录下用终端运行'学籍管理系统.py'
    \item 学籍管理系统初始密码为'python'
\end{enumerate}


\subsection{软件使用说明}
\textbf{软件内部的任何数据操作,包括增删修改信息,都会存储在内存中,当软件退出时自动保存。未正常退出时会丢失已进行的操作}\\
% 如何运行本软件,使用注意事项,每步的输入和运行结果。这部分需要截图,反映软件功能。
\text{初始界面:
输入密码时不显示输入密码(import getpass)。
密码输入错时,提示重新输入。
密码正确,提示登陆成功,等待1秒后登陆。
初始密码为'python'}

\begin{center}
    \includegraphics[width = 4in]{src/屏幕截图(13).png}
\end{center}

\text{提示用户输入数字以选择所需的功能}

\begin{center}
    \includegraphics[width = 4in]{src/屏幕截图(14).png}
\end{center}

\text{提示用户输入数字以选择展示成绩的方式}

\begin{center}
    \includegraphics[width = 4in]{src/屏幕截图(15).png}
\end{center}

\text{升序显示学生成绩}
\begin{center}
    \includegraphics[width = 4in]{src/屏幕截图(16).png}
\end{center}

\text{搜索学生姓名显示学生成绩:
未搜索到学生姓名时,提示未搜索到,按任意键返回上一菜单。
搜索到学生时,显示学生成绩。}
\begin{center}
    \includegraphics[width = 4in]{src/屏幕截图(17).png}
\end{center}

\text{搜索学生学号显示学生成绩:
未搜索到学生学号时,提示未搜索到,按任意键返回上一菜单。
搜索到学生时,显示学生成绩。}
\begin{center}
    \includegraphics[width = 4in]{src/屏幕截图(18).png}
\end{center}

\text{搜索学生姓名修改学生成绩:
未搜索到学生姓名时,提示未搜索到,按任意键返回上一菜单。
搜索到学生时,显示学生成绩,输入修改的成绩,并返回修改后的信息。}
\begin{center}
    \includegraphics[width = 4in]{src/屏幕截图(19).png}
\end{center}
\begin{center}
    \includegraphics[width = 4in]{src/屏幕截图(20).png}
\end{center}

\text{添加学生姓名时先搜索学生信息:
未搜索到学生姓名时,等待输入学生信息。
搜索到学生时,显示学生已登记信息,按任意键返回上一菜单}

\begin{center}
    \includegraphics[width = 4in]{src/屏幕截图(21).png}
\end{center}

\text{删除学生姓名时先搜索学生信息:
未搜索到学生姓名时,提示未找到学生信息。
搜索到学生时,确认是否删除。否:返回;是:删除信息,按任意键返回}


\begin{center}
    \includegraphics[width = 4in]{src/屏幕截图(22).png}
\end{center}

\text{学生成绩分析:
选择饼图和柱状图时,会生成图片,当关掉图片时自动返回上一个菜单。
选择文字信息时,返回平均分,最高分和最低分。按任意键返回}

\begin{center}
    \includegraphics[width = 4in]{src/屏幕截图(23).png}
\end{center}


\begin{center}
    \includegraphics[width = 4in]{src/屏幕截图(24).png}
\end{center}


\begin{center}
    \includegraphics[width = 4in]{src/屏幕截图(25).png}
\end{center}

\begin{center}
    \includegraphics[width = 4in]{src/屏幕截图(26).png}
\end{center}

\text{修改学籍管理系统登陆密码}
\begin{center}
    \includegraphics[width = 4in]{src/屏幕截图(27).png}
\end{center}

\text{展示上述操作后被删除和被增加的学生信息}
\begin{center}
    \includegraphics[width = 4in]{src/屏幕截图(28).png}
\end{center}

\newpage
\section{python源程序}
\begin{python}
import time
from getpass import getpass
from os import system
import numpy
import matplotlib.pyplot as plt

class student:
    def __init__(self,Name,Class,ID,Grade):
        self.Name = Name
        self.Class = Class
        self.ID = ID
        self.Grade = Grade
    def show(self):
        print('%s\t%s\t%s\t%d' % (self.Name, self.Class, self.ID, self.Grade))

def showlist():
    global opt
    system('cls')
    print('------------------------')
    print('西安交通大学学籍管理系统     ')
    print('------------------------')
    print('1 - 查看学生成绩')
    print('2 - 修改学生成绩')
    print('3 - 添加学生成绩')
    print('4 - 删除学生成绩')
    print('5 - 学生成绩分析')
    print('6 - 更改系统密码')
    print('7 - 退出管理系统')
    print('------------------------')
    while True:
        opt = input()
        if opt == '1' or opt == '2' or opt == '3' or \
        opt == '4' or opt == '5' or opt == '6' or opt =='7':
            break
    opt = int(opt)
#加载信息(info第一行为密码,剩余的为学生信息)
def loadinfo():
    global info
    global key
    data = open('info.txt', 'r', encoding='UTF-8').read()
    data = data.split('\n')
    key = data[0]
    data.pop(0)
    for stu in data:
        stu = stu.split('\t')
        try:
            info.append(student(stu[0],stu[1],stu[2],int(stu[3])))
        except IndexError:
            pass
#初始化
def init():
    global info
    global key
    loadinfo()
    system('cls')
    print('------------------------')
    print('西安交通大学学籍管理系统     ')
    print('------------------------')
    k = getpass('请输入管理员密码:')
    while k != key:
        system('cls')
        print('------------------------')
        print('西安交通大学学籍管理系统     ')
        print('------------------------')
        k = getpass('密码错误,重新输入:')
    print('登陆成功!')
    time.sleep(1)
#查找学生信息
def look():
    global info
    opt = 0
    while(opt != 3):
        system('cls')
        print('------------------------')
        print('西安交通大学学籍管理系统     ')
        print('------------------------')
        print('<查看学生成绩>')
        print('1 - 按成绩升序')
        print('2 - 按成绩降序')
        print('3 - 按学生姓名查找')
        print('4 - 按学生学号查找')
        print('5 - 返回')
        print('------------------------')
        while True:
            opt = input()
            if opt == '1' or opt == '2' or opt == '3' or opt == '4' or opt == '5':
                break
        opt = int(opt)
        if opt == 1:
            info.sort(key = lambda x : x.Grade)
        elif opt == 2:
            info.sort(key = lambda x : x.Grade, reverse= True)
        elif opt == 3:
            name = input('请输入学生姓名:')
            for s in info:
                if s.Name == name:
                    s.show()
                    system('pause')
                    return
            print('未找到该学生!')
            system('pause')
            return
        elif opt == 4:
            id = input('请输入学生学号:')
            for s in info:
                if s.ID == id:
                    s.show()
                    system('pause')
                    return
            print('未找到该学生!')
            system('pause')
            return
        else:
            return
        system('cls')
        for stu in info:
            stu.show()
        system('pause')
#修改学生信息
def revise():
    system('cls')
    global info
    print('------------------------')
    print('西安交通大学学籍管理系统     ')
    print('------------------------')
    print('<修改学生成绩>')
    name = input('请输入学生姓名:')
    flag = True
    for s in info:
        if s.Name == name:
            flag = False
            s.show()
            grade = int(input("请输入修改后的成绩:"))
            s.Grade = grade
            print('修改后的学生信息:')
            s.show()
            break
    if flag == True:
        print('未找到该学生!')
    system('pause')
#添加学生信息
def add():
    system('cls')
    global info
    print('------------------------')
    print('西安交通大学学籍管理系统     ')
    print('------------------------')
    print('<添加学生成绩>')
    Name = input('请输入学生姓名:')
    Class = input('请输入学生班级:')
    ID = input('请输入学生学号:')
    Grade = int(input('请输入学生成绩:'))
    for s in info:
        if s.Name == Name:
            print('该学生成绩已经登记!')
            system('pause')
            return
    info.append(student(Name,Class,ID,Grade))
    print('添加的学生成绩信息如下:')
    student(Name, Class, ID, Grade).show()
    system('pause')
#删除学生信息
def delete():
    system('cls')
    global info
    print('------------------------')
    print('西安交通大学学籍管理系统     ')
    print('------------------------')
    print('<删除学生成绩>')
    Name = input('请输入学生姓名:')
    for s in info:
        if s.Name == Name:
            print('确认删除此学生记录?(Y/N)')
            ans = input()
            if ans == 'Y' or ans == 'y':
                info.remove(s)
                print('已删除')
                system('pause')
                return
            print('已取消删除操作')
            system('pause')
            return
    print('未找到学生信息')
    system('pause')
    return
#离开(保存文件)
def leave():
    system('cls')
    print('------------------------')
    print('西安交通大学学籍管理系统     ')
    print('------------------------')
    print('<退出管理系统>')
    with open('info.txt','w', encoding='utf-8') as file:
        file.write(key)
        file.write('\n')
        for s in info:
            ss = s.Name + '\t' + s.Class + '\t' + s.ID + '\t' + str(s.Grade) + '\n'
            file.write(ss)
    system('cls')
#更改密码
def change():
    global key
    system('cls')
    print('------------------------')
    print('西安交通大学学籍管理系统     ')
    print('------------------------')
    print('<更改系统密码>')
    cert = getpass('请输入原密码:')
    if cert != key:
        print('密码错误,更改密码失败')
        system('pause')
        return
    cert = getpass('请输入新密码:')
    key = cert
    print('成功修改密码')
    system('pause')
#学生成绩分析
def analyse():
    global key
    system('cls')
    print('------------------------')
    print('西安交通大学学籍管理系统     ')
    print('------------------------')
    print('<学生成绩分析>')
    print('1 - 饼图')
    print('2 - 柱状图')
    print('3 - 文字信息')
    print('4 - 返回')
    print('------------------------')

    sizes = [0,0,0,0,0]
    for s in info:
        if s.Grade >= 90:
            sizes[0] += 1
        elif s.Grade >= 80:
            sizes[1] += 1
        elif s.Grade >= 70:
            sizes[2] += 1
        elif s.Grade >= 60:
            sizes[3] += 1
        else:
            sizes[4] += 1

    labels = '90-100', '80-89', '70-79', '60-69', '0-59'
    colors = 'yellowgreen', 'gold', 'lightskyblue', 'lightcoral', 'pink'
    explode = 0, 0, 0, 0.1, 0
    plt.rcParams['font.sans-serif'] = ['SimHei']

    opt = int(input())
    if opt == 1:
        plt.pie(sizes, explode=explode, labels=labels, colors=colors, autopct='%2.2f%%', shadow=True, startangle=30)
        plt.axis('equal')
        plt.title('2015级西安交通大学学生平均成绩分析\n')
        plt.show()
    elif opt == 2:
        plt.bar(labels, sizes, color='g', width=0.2)
        plt.title('2015级西安交通大学学生平均成绩分析')
        plt.yticks(sizes)
        plt.show()
    elif opt == 3:
        sum = 0
        m = n = info[0]
        for s in info:
            if s.Grade > m.Grade:
                m = s
            if s.Grade < n.Grade:
                n = s
            sum += s.Grade
        aver = sum / len(info)
        print('平均成绩为:%.2f'%aver)
        print('最高分:%s %.1f分'%(m.Name,m.Grade))
        print('最低分:%s %.1f分' % (n.Name, n.Grade))
        print('------------------------')
        system('pause')
    else:
        return

if __name__ == '__main__':
    opt = 0
    key = ''
    info = []
    init()
    showlist()
    while(opt != 7):
        system('cls')
        if opt == 1:
            look()
        elif opt == 2:
            revise()
        elif opt == 3:
            add()
        elif opt == 4:
            delete()
        elif opt == 5:
            analyse()
        elif opt == 6:
            change()
        else:
            break
        showlist()
    leave()

\end{python}

\newpage
\section{实验总结}

通过这次python选修课程的大作业,我学会了用python语言进行程序设计,这次的大作业仅仅是这个课程所学内容的一个部分,主要是展示了python程序设计语言的基本使用语法,实现了python基于命令行的学籍管理系统。我觉得这门课的意义是引导与入门,在比较短的时间内让我们基本上掌握python这个工具,了解基本的库,学习使用方法,为我们将来的学习工作带来效率的提高。这次的实验不是基于算法设计的一个程序,更强调的是实现一个基本的业务逻辑,用来提高python的熟悉程度。通过这次实验,我感触最多的就是python的强大和便捷:仅仅一个晚上就能快速实现自己想要实现的东西,大大简化了操作的复杂度。同时,python强大的库函数也吸引着我去尝试用python解决学习生活中的问题。这次课程还引入了网络,多线程等内容,让我能更全面的接触python,使用python,爱上python。最后感谢任课老师地指导,让这一门选修课变得充实而有意义。

\end{document}